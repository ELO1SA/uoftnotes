% Meta data
\documentclass[oneside, 12pt]{article}
\usepackage[utf8x]{inputenc}
\usepackage[english]{babel}
\usepackage{amsfonts}
\usepackage{amsmath}
\usepackage{amssymb}
\usepackage{url}
\usepackage[pdfencoding=auto, psdextra]{hyperref}
\usepackage{fancyhdr} 
\usepackage{youngtab}

% customized commands
\newcommand{\settag}[1]{\renewcommand{\theenumi}{#1}}
\newcommand{\R}{\mathbb{R}}
\newcommand{\real}{\mathbb{R}}
\newcommand{\complex}{\mathbb{C}}
\newcommand{\field}{\mathbb{F}}
\newcommand{\double}[1]{\mathbb{#1}} % Set to behave like that on word
\newcommand{\qed}{\hfill $\mathcal{Q}.\mathcal{E}.\mathcal{D}.\dagger$}
\newcommand{\tbf}[1]{\textbf{#1}}
\newcommand{\tit}[1]{\textit{#1}}
\newcommand{\contradiction}{$\longrightarrow\!\longleftarrow$}
\newcommand{\overbar}[1]{\mkern 1.5mu\overline{\mkern-1.5mu#1\mkern-1.5mu}\mkern 1.5mu}
\newcommand{\proof}{\tit{\underline{Proof:}}} % This equivalent to the \begin{proof}\end{proof} block
\newcommand{\proofforward}{\tit{\underline{Proof($\itemmmplies$):}}}
\newcommand{\proofback}{\tit{\underline{Proof($\itemmmpliedby$):}}}
\newcommand{\proofsuperset}{\tit{\underline{Proof($\supseteq$):}}}
\newcommand{\proofsubset}{\tit{\underline{Proof($\subseteq$):}}}
\newcommand{\trans}[3]{$#1:#2\rightarrow{}#3$}
\newcommand{\map}[3]{\text{$\left[#1\right]_{#2}^{#3}$}}
\newcommand{\dime}[1]{\text{dim}(#1)}
\newcommand{\mat}[2]{M_{#1 \times #2}(\R)}
\newcommand{\aug}{\fboxsep=-\fboxrule\!\!\!\fbox{\strut}\!\!\!}
\newcommand{\basecase}{\textsc{\underline{Basis Case:}} }
\newcommand{\itemmnductive}{\textsc{\underline{Inductive Step:}} }
\newcommand{\norm}[1]{\left\lVert#1\right\rVert}
% tags that start w/ type setter fonts
\newcommand{\itemm}[1]{\item \texttt{#1}}
% Call settag{\ldots} first to initialize, and then \para{} for a new paragraph
\newcommand{\para}[1]{\item \tbf{#1}}
\newcommand{\va}{\mathbf{a}}
\newcommand{\vb}{\mathbf{b}}
\newcommand{\vv}{\mathbf{v}}
\newcommand{\vu}{\mathbf{u}}
\newcommand{\vw}{\mathbf{w}}
\newcommand{\vx}{\mathbf{x}}
\newcommand{\ve}{\mathbf{e}}
\newcommand{\vy}{\mathbf{y}}
\newcommand{\vz}{\mathbf{z}}
\newcommand{\vzero}{\mathbf{0}}
% For convenience, I am setting both of these to refer to the same thing.
\newcommand{\ba}{\mathbf{a}}
\newcommand{\bb}{\mathbf{b}}
\newcommand{\bv}{\mathbf{v}}
\newcommand{\bu}{\mathbf{u}}
\newcommand{\bw}{\mathbf{w}}
\newcommand{\bx}{\mathbf{x}}
\newcommand{\be}{\mathbf{e}}
\newcommand{\by}{\mathbf{y}}
\newcommand{\bzero}{\mathbf{0}}


\pdfinfo{
   /Author (Tingfeng Xia)
   /Title  (HTML tag manuel)
   /CreationDate (D:20190531)
   /Subject (Web Development)
}

\title{%
  \textbf{Web Development - HTML language}\\
  \large tags maunel}
\author{Tingfeng Xia}
\date{2019.05}

\begin{document}


\maketitle
\newpage % make a new page for actual contents, possibly a content table
\mbox{}
\vfill
\noindent by Tingfeng Xia \\


\noindent HyperText Markup Language, commonly referred to as HTML, 
is the standard markup language used to create web pages. Although this langugage itself
is not hard to learn, the huge amount of tags that where involved in the language usually make
new learners feel overwhelmed at first contact. This manuel serves the pure purpose as a 
look up manuel for all the tags. We will proceed alphabetically.
\newpage

\begin{itemize}
	\itemm{<!-- ... -->} is the comment tag, the \texttt{...} inside is the comment
	message. Notice that this supports multiline comments.
	\itemm{<!DOCTYPE>} Ususally we put \texttt{<!DOCTYPE html>} as the first line in a html file
	to tell the browser to use html 5 as the language. Do notice that browsers are indifferent
	to the capitalization, but we do this to follow the convention.
	\itemm{<a>} is the syntax for a hyper-reference, for example we can write 
	\texttt{<a href="/path/to/something">go to something</a>} and this will generate a hyper-linked
	text ``go to something'' which will take you to that something. Similar to the way it is
	in a Microsoft Office, the hyper-link has color to different its status: blue for not-clicked
	and purple for already accessed. 
	\itemm{<abbr>} defines an abbreviation. You can write, for example, 
	\texttt{<abbr title="university of Toronto">UofT</abbr>}, which will display \, ``UofT'' on the
	outside and when you hover your mouse over the word, the full name will appear.
	\itemm{<address>} defines contact information. The typical uses (by convention) of the address tag are:
		\begin{enumerate}
			\item If the \texttt{<address>} tag is inside the body element, then it provides the info
			for contacting the owner/writer of the entire document.
			\item If the \texttt{<address>} tag is inside a article element, then it provides the info
			for contacting the owner/writer of that particular article section.
			\item Otherwise, the \texttt{<address>} tag is usually in the ``other'' part of footer 
			section of the page.
		\end{enumerate}
		notice that by very strong convention we never use address tag directly for an address, unless 
		the address is one part of a bigger family of information.
	\itemm{<area>} defines area inside a click-able image, which means area is always inside a map tag.
	We can use area for hyperlinks, for example \texttt{<area shape="rect/circle" coords="???" alt="??" href="?.htm">}, 
	which will direct us to the \texttt{?.htm} refer-ed upon clicked inside the cords for that particular shape
	\itemm{<article>} defines an area for article. This is particularly useful for posts in a BBS, articles
	in a blog, news or stories and comments that users leave on a certain site.
	\itemm{<aside>} defines a supplementary section for the article section (defined above). And
	by very strong convention, the stuff inside the aside tag has to have connection with the near-by
	article. In day to day use, this could be used as the side column (for annotations, comments, etc).
	\itemm{<audio>} is for audio. The audio element supports MP3, Wav and Ogg. Notice that not every browser
	supports such tag, so may wish put text inside this tag and those text will be displayed in cases that
	the browser doesn't support the audio tag.
	\itemm{<b>} stands for bold, one of the most commonly used tags if not the most common tag used.
	\itemm{base} specifies, for thee page, the default absolute home path that will be used and the
	default method that a new tab/page window will open upon clicking a hyper-refed link. For example,
	as mentioned before, setting \texttt{target=\_blank} will tell the browser to open a new tab for
	newly opened page. We shall put this inside the head section at the first line (and can only be
	used once), so that the rest of the head section could benefit from this default setting.
	\itemm{<bdi>} is called bi-directional isolation. This allows text inside this tag to be different
	direction than the parent section. This may sound silly at first, but do remember that not all language
	writes from left to right, for example Hebrew writes from right to left. But if that is the case,
	your left and right is swapped inside these characters while they are still what used to be outside, and
	this could confuse the compiler a lot. A typical use case would be a comment area that is designed
	for multi-nation access where you have absolutely no control over what the users will post where it
	would be a good idea to isolate each comment and allow their natural writing orientation. 
	\itemm{<bdo>} is called bi-directional override. It has one required attribute \texttt{dir} which
	we shall set to either \texttt{ltr} (left to right) or \texttt{rtl} (right to left) to specify the
	direction that we want the text to appear in. 
	\itemm{<big>} is for a bigger text size.
	\itemm{<blockquote>} quotes a section of text which the browser typically indents a bit. You can
	also set the \texttt{cite} attribute to mark the source of the information, but that will not be
	displayed. 
	\itemm{<body>} is one of the required tags for a html document, it specifies the body section of
	the page, which is usually where the interesting stuff happens on a page lives.
	\itemm{<br>} is an empty tag (meaning that it has no closing tag) that stands for break (for a new line).
	\itemm{<button>} marks a button on a page.
	\itemm{<canvas>} is a container section, meaning itself has no meaning while it relies on JavaScript
	to ``draw'' on this canvas. Of course, you can still set the size of the canvas using attribute, 
	not necessarily entirely rely on Scripts.
	\itemm{<caption>} is the title for a table and oddly has to come after the table section. You can have
	one (or none) caption per tableau. 
	\itemm{<center>} is a deprecated, but yet somehow quite useful tag that forces the text to be
	centred in a line (or as a section). 
	\itemm{<cite>} defines the title for, for example, books, songs, albums, televisions and sculptures, etc.
	Do notice that name for the creator is not considered as an element of the title.
	\itemm{<code>} is a code block, similar to the texttt command in \LaTeX.
	\itemm{<col>} 
\end{itemize}


\end{document}
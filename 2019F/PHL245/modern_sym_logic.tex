\documentclass[10pt]{article}
\usepackage[margin=3.5cm]{geometry}
\usepackage[utf8x]{inputenc}
\usepackage[english]{babel}
\usepackage{amsfonts}
\usepackage{amsmath}
\usepackage{amssymb}
\usepackage{url}
\usepackage{bookmark}
\usepackage{hyperref}
\usepackage{fancyhdr} 
\usepackage{youngtab}
\usepackage{logicproof}
\usepackage[normalem]{ulem}
\usepackage[
    type={CC},
    modifier={by-nc-sa},
    version={4.0},
]{doclicense}

% customized commands
\newcommand{\settag}[1]{\renewcommand{\theenumi}{#1}}
\newcommand{\R}{\mathbb{R}}
\newcommand{\real}{\mathbb{R}}
\newcommand{\complex}{\mathbb{C}}
\newcommand{\field}{\mathbb{F}}
\newcommand{\double}[1]{\mathbb{#1}} % Set to behave like that on word
\newcommand{\qed}{\hfill $\mathcal{Q}.\mathcal{E}.\mathcal{D}.\dagger$}
\newcommand{\tbf}[1]{\textbf{#1}}
\newcommand{\tit}[1]{\textit{#1}}
\newcommand{\contradiction}{$\longrightarrow\!\longleftarrow$}
\newcommand{\overbar}[1]{\mkern 1.5mu\overline{\mkern-1.5mu#1\mkern-1.5mu}\mkern 1.5mu}
\newcommand{\proof}{\tit{\underline{Proof:}}} % This equivalent to the \begin{proof}\end{proof} block
\newcommand{\proofforward}{\tit{\underline{Proof($\implies$):}}}
\newcommand{\proofback}{\tit{\underline{Proof($\impliedby$):}}}
\newcommand{\proofsuperset}{\tit{\underline{Proof($\supseteq$):}}}
\newcommand{\proofsubset}{\tit{\underline{Proof($\subseteq$):}}}
\newcommand{\trans}[3]{$#1:#2\rightarrow{}#3$}
\newcommand{\map}[3]{\text{$\left[#1\right]_{#2}^{#3}$}}
\newcommand{\dime}[1]{\text{dim}(#1)}
\newcommand{\mat}[2]{M_{#1 \times #2}(\R)}
\newcommand{\aug}{\fboxsep=-\fboxrule\!\!\!\fbox{\strut}\!\!\!}
\newcommand{\basecase}{\textsc{\underline{Basis Case:}} }
\newcommand{\inductive}{\textsc{\underline{Inductive Step:}} }
\newcommand{\norm}[1]{\left\lVert#1\right\rVert}
% Call settag{\ldots} first to initialize, and then \para{} for a new paragraph
\newcommand{\va}{\mathbf{a}}
\newcommand{\vb}{\mathbf{b}}
\newcommand{\vv}{\mathbf{v}}
\newcommand{\vu}{\mathbf{u}}
\newcommand{\vw}{\mathbf{w}}
\newcommand{\vx}{\mathbf{x}}
\newcommand{\ve}{\mathbf{e}}
\newcommand{\vy}{\mathbf{y}}
\newcommand{\vz}{\mathbf{z}}
\newcommand{\vc}{\mathbf{c}}
\newcommand{\vm}{\mathbf{m}}
\newcommand{\vh}{\mathbf{h}}
\newcommand{\vzero}{\mathbf{0}}
% For convenience, I am setting both of these to refer to the same thing.
\newcommand{\ba}{\mathbf{a}}
\newcommand{\bb}{\mathbf{b}}
\newcommand{\bv}{\mathbf{v}}
\newcommand{\bu}{\mathbf{u}}
\newcommand{\bw}{\mathbf{w}}
\newcommand{\bx}{\mathbf{x}}
\newcommand{\be}{\mathbf{e}}
\newcommand{\by}{\mathbf{y}}
\newcommand{\bzero}{\mathbf{0}}
\newcommand{\boldf}{\mathbf{f}}
\newcommand{\bg}{\mathbf{g}}
\newcommand{\bm}{\mathbf{m}}
\renewcommand{\iff}{\leftrightarrow}

\title{PHL245 Modern Symbolic Logic}
\author{\ccLogo \,\,Tingfeng Xia}
\date{Fall 2019, modified on \today}

\begin{document}
\maketitle
\doclicenseThis
\tableofcontents
\newpage

\section{Arguments}
\subsection{Validity}
We say a deductive argument is valid iff it is not invalid. This means that we can
find out if a argument is valid or not by assessing the possiblity of the case where 
the premise is true and conclusion is false.
\subsection{Soundness}
We say a deductive argument is sound iff it is 
\begin{itemize}
    \item It is valid
    \item All the premises are TRUE
\end{itemize}

\section{Semantics in Sentential Logic}
\subsection{Syntax}
\paragraph{Sentential Logic(SL)} Complex (compound) statements are all built up by joining statements together using LOGICAL CONNECTIONS.
we have $\wedge, \vee, \rightarrow, \leftarrow, \sim$. Where $\sim$ is the only unitary connector and others are binary.

\paragraph{Atomic vs Molecular} A \textbf{statement} is Atomic if it has no logical connector and is molecular otherwise. We use \textbf{P-Z letters} to represent \textbf{atomic statments}
Here is an example: You can have fries or salad. $\equiv P \vee Q.$ Then, $P$ is ``You can have fires'' and $Q$ is ``You can have salad''. Notice that we are folowing the definition that $P$ and $Q$ are \textbf{statments}.

\paragraph{Informal Notation Hierachy}\footnote{We use this Informal notation because the formal one is cumbersome. But in order to use the informal one, we have some conventions to follow.} 
We shall see this through some examples:
\begin{itemize}
    \item In official notation, we need $(P\vee Q)$ while it is safe to drop the parenthesis.
\end{itemize}
We can check if a statment is official or informal if it has the same number of brackets as binary connectors in the statement. If you see a bracket around a unitary connector, the sentence is not well-formed.
\textbf{Right Most Rule} We say the rightmost connector in a sentence with connectors of the same level the main connector.



\section{Truth Tables}
\subsection{Summary} First we remark that we are here talking about binary operators, 
so they will have and only have two operands. We have the following to consider:
\begin{itemize}
    \item \textbf{Logical OR ($\vee$)} is true when either one of the operands is true. 
        It evaluates to false otherwise.
    \item \textbf{Logical AND ($\land$)} is true when both of the operands are true and false otherwise.
    \item \textbf{Implication ($\implies$)} is true in two cases. The first case is when the
        first operand is false and second case is when both operands are true.
    \item \textbf{Double implication, Iff ($\iff$)} is true when either both operands are true 
        or when both operands are false.
    \item \textbf{Negation ($\neg$\footnote{Or in this course, we may see $\sim$})} is true when the operand
        is false and false otherwise.
\end{itemize}

\subsection{Sementic Properties in Truth Tables}
\begin{itemize}
    \item Tautology if always true
    \item Contradiction if always false 
    \item Contingent if mixed
    \item Consistent if there exists a row where the conclusion is all true
    \item Inconsistent if all rows are not all true, negation of the consistent
    \item Logically equivalent if two sentences have the same truth table
\end{itemize}



\subsection{Full truth tables}
Example: \footnote{This example was adapted from Scharer 4.4 EG3.}
evaluate $(P\land \neg Q)\vee R. ~~\neg R\vee Q. ~~\therefore \neg P\implies Q$. 
Notice that this is equivalent to evaluating
\begin{equation*}
    \left(\left(\left(P\land \neg Q\right)\vee R \right) \land 
    \left(\neg R\vee Q\right)\right) \implies \neg P\implies Q
\end{equation*}
% and by staring at it we see this statement is valid and only valid
%  when $(P,Q,R)\in \{(T,T,T),(T,F,F),(F,T,T)\}$, which has a non-null 
%  set of solution. Hence the statement is consistent.



\section{Derivations in Sentential Logic}
\subsection{Basic Derivations}
Full truth tables are impratical when we have a long sentence to check validity 
and it turns out that derivation is a nice and simple way to achieve such goal.
% \subsubsection{Derivation Rules}
\subsubsection{Modus Ponens (MP)} $\phi \rightarrow \psi$, if we know $\phi$, then it is reasonable to say $\psi$
\subsubsection{Modus Tollens (MT)} $\phi \rightarrow \psi$ then $\sim\psi \rightarrow \sim\phi$, aka contrapositive
\subsubsection{Fallacys} If we deny the antecedent or affirm the consequent, that \textit{doesn't} tell us anything!
\subsubsection{Double Negation (DN)} $\sim\sim\phi \equiv \phi$. Negating a statement two times is the same as the statement itself.
\subsubsection{Repetition (R)} $\phi \rightarrow \phi$ If you know $\phi$ then you know $\phi$
\subsubsection{Examples of rules above}
Let's see the Modus Tollens in action
\begin{equation*}
    \sim (S\vee P) \rightarrow \sim R.~ R. ~\text{so}~ S\vee P
\end{equation*}
THIS IS INCOORECT! We have to take the rules ``as is'', since we are constructing a
contrapositive, we will have $\sim\sim (S\vee P)$ as our conclusion. Here is an example of a 
correct usage of Modus Tollens
\begin{equation*}
    \sim P \rightarrow \sim (S\rightarrow Z). ~ \sim\sim (S\rightarrow Z) ~ so ~ \sim\sim P
\end{equation*}
\textbf{Take away:} We can only apply rules ``as they are'' so the double negation, for example,
only works for statements that are of the form $\sim\sim(\text{complicated stuff})$ \textit{we can't 
jump over any step, we have to write down all the steps.}

\subsubsection{Justification Types}
There are three things that we can do here: 1.Restate any of the given premises. 
2. Use any of the aforementioned rules (write down the lines that we used the rules on and 
the abbreviation of the rule that we used) 3. Direct Derivation: usually an 
indicator of arriving at the final conclusion.

\subsubsection{Example (S3.3 E2a)}
Consider the problem $P\rightarrow Q. ~ R\rightarrow\sim Q.~ \sim S \rightarrow R.~ P.~ \text{so}~ S.$
\begin{logicproof}{1}
    \text{\sout{show}}~ S \\
    \begin{subproof}
        P\rightarrow Q & Pr1 \\
        P & Pr4 \\
        Q & 2,3,MP \\
        R \rightarrow \sim Q & Pr2 \\
        \sim\sim Q & 4,DN \\
        \sim R & 5,6,MT \\
        \sim S\rightarrow R & Pr3 \\
        \sim\sim S & 7,8,MT \\
        S & 9,DN \\
        & 10,DD 
    \end{subproof}
    <neglect~this~line>
\end{logicproof}

\subsubsection{Example (S3.3 E2b)}
Consider the problem $Y. ~ X\rightarrow(Y\rightarrow Z).~ \sim X\rightarrow \sim W. ~ W. ~\text{so}~ \sim\sim Z$
\begin{logicproof}{1}
    \text{\sout{show}}~ \sim\sim Z  & \text{show conclusion}\\
    \begin{subproof}
        \sim\sim W & Pr4, DN \\
        \sim X \rightarrow \sim W & Pr3\\
        \sim\sim X & 2,3,MT \\
        X & 4, DN \\
        Y \rightarrow Z & 5, Pr2, MP \\
        Y & Pr1 \\
        Z & 6,7,MP \\
        \sim\sim Z & 8,DN \\
        & 9,DD
    \end{subproof}
    <neglect~this~line>
\end{logicproof}

\subsubsection{Available Lines}
In doing a proof, we want to keep track of what is available for us to use:
\begin{itemize}
    \item Premises, we can always use the premises
    \item Unboxed lines that is not a show line
    \item N.B. A crossed unboxed show line is available (something that we have already shown)
\end{itemize}

\subsubsection{Completeness of a Derivation}
\begin{itemize}
    \item Every show line is crossed off (goals complete)
    \item All lines that are not show lines are boxed off (subproof complete)
    \item Every line except show lines is properly justified (state the reasons for derivations of each step)
\end{itemize}

\subsubsection{Abbreviations}
\begin{itemize}
    \item Donnot restate the premises
    \item Do multiple moves in one line
\end{itemize}
Here is an example of using the abbrevaitions in action. Consider the exmaple
$(P\rightarrow \sim Q) \rightarrow(\sim R \rightarrow S). \sim S. \sim(P\rightarrow \sim Q)\rightarrow T. T\rightarrow S ~\text{so}~R$
\begin{logicproof}{1}
    \text{\sout{show}}~ \sim\sim Z  & \text{show conclusion} \\
    \begin{subproof}
        \sim T & P2, P4, MT \\
        P \rightarrow \sim Q & 2, P3, MT, DN \\
        \sim R \rightarrow S & 3, P1, MP \\
        R & 4, P2, MT, DN, DD
    \end{subproof}
    <neglect~this~line>
\end{logicproof}

\subsubsection{Conditional Derivation (Hypothetical Reasoning)}
We have seen above are all direct derivations, there are two more types of derivations
common. They are Conditional Derivation and Indirect Derivation. Lets take a look at conditional
derivation here. The general format of this type of proofs is $\phi\rightarrow\psi$. To prove 
a conditional, we assume $\phi$ and then prve that $\psi$ follows.

\paragraph{Example} Consider the problem $T\rightarrow S. \sim T \rightarrow \sim R. \text{so}~ R\rightarrow S$
\begin{logicproof}{2}
    \text{\sout{show}}~ R\rightarrow S\\
    \begin{subproof}
        R & ACD (\text{Assume antecedent}) \\
        \text{\sout{show}}~ S  & \text{show consequence}\\
        \begin{subproof}
            \sim\sim R & 2, DN \\
            T & 4, P2, MT, DN \\
            S & P1, 5, MP, DD
        \end{subproof}
        & 3, CD(Conditional Derivation)
    \end{subproof}
    <neglect~this~line>
\end{logicproof}
\paragraph{Caveat:} It is not required to write the derivation
 as a subproof, we can also
carry the moves in just one layer and state on which line we 
achived a conditional derivation

\subsubsection{Indirect Derivation (Reductio ad Absurdum)}
The question takes the following form ``Show $\phi$''. If we assume 
$\sim \phi$ and derive a contradiction, then our assumption must be false.
\paragraph{Example} Consider the problem $P\rightarrow \sim Q. 
R\rightarrow Q. \sim R \rightarrow \sim P. ~\text{so}~ P$
\begin{logicproof}{1}
    \text{\sout{show}} \sim P \\
    \begin{subproof}
        P & AID(Assume ID) \\
        \sim Q & 2, P1, MP \\
        \sim R & P2, 3, MT \\
        \sim P & 4, P3, MP \\
        & 2, 5, ID (Indirect Derivation / Reductio ad Absurdum)
    \end{subproof}
    <neglect~this~line>
\end{logicproof}

\subsubsection{Breaking down show lines}
\begin{itemize}
    \item Look at your most recent show line
    \item If it's of the form $\phi\rightarrow \psi$, start a CD 
    \item If it's any other form, start an ID
\end{itemize}
This works because our system allows mixed derivations!

\subsection{Ten Basic Rules}
\paragraph{Types of Rules}
\begin{itemize}
    \item \textbf{Elimination Rule} remove the connective (Elimination
     Rules are Automatic Moves)
    \item \textbf{Introduction} introduce the connective
\end{itemize}

\subsubsection{Conjunction}
\paragraph{Simplification (S or SL/SR)} If I know $\phi\land \psi$ then
 I know $\phi$ and I know $\psi$
\paragraph{Adjunction (ADJ)} If know $\phi$ and I know $\psi$ then I
 know $\phi\land\psi$

\subsubsection{Disjunction}
\paragraph{Modus Tollendo Ponens (MTP)} If I know $\phi \vee \psi$ and 
I know $\sim\phi$ then I know $\psi$. WLOG the other way around is also coorect.
\paragraph{Addition (ADD)} If I know $\phi$ then  I know $\phi \vee \psi$.
 WLOG the other way around is also true.

\subsubsection{Biconditional}
\paragraph{Biconditional-Conditional} If I know $\phi \iff \psi$ then I 
know that $\phi \rightarrow \psi$ and $\psi \rightarrow \phi$
\paragraph{Conditional-Biconditional} If I knwo $\phi \rightarrow \psi$ 
and $\psi \rightarrow \phi$, then (by definition) I know  $\phi \iff \psi$

% \subsubsection{Contradiction Generators}


\subsection{Theorems}
\textit{Definition:} A Theorem is a tautology. And being a tautology 
means it could be derived from any set of premises, even including
 $\emptyset$ as the premise set.




\end{document}

\documentclass[11pt]{article}

\usepackage[utf8x]{inputenc}
\usepackage[english]{babel}
\usepackage{amsfonts}
\usepackage{amsmath}
\usepackage{amssymb}
\usepackage{url}
\usepackage{bookmark}
\usepackage{hyperref}
\usepackage{fancyhdr} 
\usepackage{youngtab}
\usepackage[
    type={CC},
    modifier={by-nc-sa},
    version={4.0},
]{doclicense}

% customized commands
\newcommand{\settag}[1]{\renewcommand{\theenumi}{#1}}
\newcommand{\R}{\mathbb{R}}
\newcommand{\real}{\mathbb{R}}
\newcommand{\complex}{\mathbb{C}}
\newcommand{\field}{\mathbb{F}}
\newcommand{\double}[1]{\mathbb{#1}} % Set to behave like that on word
\newcommand{\qed}{\hfill $\mathcal{Q}.\mathcal{E}.\mathcal{D}.\dagger$}
\newcommand{\tbf}[1]{\textbf{#1}}
\newcommand{\tit}[1]{\textit{#1}}
\newcommand{\contradiction}{$\longrightarrow\!\longleftarrow$}
\newcommand{\overbar}[1]{\mkern 1.5mu\overline{\mkern-1.5mu#1\mkern-1.5mu}\mkern 1.5mu}
\newcommand{\proof}{\tit{\underline{Proof:}}} % This equivalent to the \begin{proof}\end{proof} block
\newcommand{\proofforward}{\tit{\underline{Proof($\implies$):}}}
\newcommand{\proofback}{\tit{\underline{Proof($\impliedby$):}}}
\newcommand{\proofsuperset}{\tit{\underline{Proof($\supseteq$):}}}
\newcommand{\proofsubset}{\tit{\underline{Proof($\subseteq$):}}}
\newcommand{\trans}[3]{$#1:#2\rightarrow{}#3$}
\newcommand{\map}[3]{\text{$\left[#1\right]_{#2}^{#3}$}}
\newcommand{\dime}[1]{\text{dim}(#1)}
\newcommand{\mat}[2]{M_{#1 \times #2}(\R)}
\newcommand{\aug}{\fboxsep=-\fboxrule\!\!\!\fbox{\strut}\!\!\!}
\newcommand{\basecase}{\textsc{\underline{Basis Case:}} }
\newcommand{\inductive}{\textsc{\underline{Inductive Step:}} }
\newcommand{\norm}[1]{\left\lVert#1\right\rVert}
% Call settag{\ldots} first to initialize, and then \para{} for a new paragraph
\newcommand{\va}{\mathbf{a}}
\newcommand{\vb}{\mathbf{b}}
\newcommand{\vv}{\mathbf{v}}
\newcommand{\vu}{\mathbf{u}}
\newcommand{\vw}{\mathbf{w}}
\newcommand{\vx}{\mathbf{x}}
\newcommand{\ve}{\mathbf{e}}
\newcommand{\vy}{\mathbf{y}}
\newcommand{\vz}{\mathbf{z}}
\newcommand{\vc}{\mathbf{c}}
\newcommand{\vm}{\mathbf{m}}
\newcommand{\vh}{\mathbf{h}}
\newcommand{\vzero}{\mathbf{0}}
% For convenience, I am setting both of these to refer to the same thing.
\newcommand{\ba}{\mathbf{a}}
\newcommand{\bb}{\mathbf{b}}
\newcommand{\bv}{\mathbf{v}}
\newcommand{\bu}{\mathbf{u}}
\newcommand{\bw}{\mathbf{w}}
\newcommand{\bx}{\mathbf{x}}
\newcommand{\be}{\mathbf{e}}
\newcommand{\by}{\mathbf{y}}
\newcommand{\bzero}{\mathbf{0}}
\newcommand{\boldf}{\mathbf{f}}
\newcommand{\bg}{\mathbf{g}}
\newcommand{\bm}{\mathbf{m}}

\title{PHL245 Modern Symbolic Logic}
\author{\ccLogo \,\,Tingfeng Xia}
\date{Fall 2019, modified on \today}

\begin{document}
\maketitle
\doclicenseThis
\tableofcontents
\newpage

\section{Arguments}
\subsection{Validity}
We say a deductive argument is valid iff it is not invalid. This means that we can
find out if a argument is valid or not by assessing the possiblity of the case where 
the premise is true and conclusion is false.
\subsection{Soundness}
We say a deductive argument is sound iff it is 
\begin{itemize}
    \item It is valid
    \item All the premises are TRUE
\end{itemize}

\section{Semantics in Sentential Logic}
\subsection{Syntax}
\paragraph{Sentential Logic(SL)} Complex (compound) statements are all built up by joining statements together using LOGICAL CONNECTIONS.
we have $\wedge, \vee, \rightarrow, \leftarrow, \sim$. Where $\sim$ is the only unitary connector and others are binary.

\paragraph{Atomic vs Molecular} A \textbf{statement} is Atomic if it has no logical connector and is molecular otherwise. We use \textbf{P-Z letters} to represent \textbf{atomic statments}
Here is an example: You can have fries or salad. $\equiv P \vee Q.$ Then, $P$ is ``You can have fires'' and $Q$ is ``You can have salad''. Notice that we are folowing the definition that $P$ and $Q$ are \textbf{statments}.

\paragraph{Informal Notation Hierachy}\footnote{We use this Informal notation because the formal one is cumbersome. But in order to use the informal one, we have some conventions to follow.} 
We shall see this through some examples:
\begin{itemize}
    \item In official notation, we need $(P\vee Q)$ while it is safe to drop the parenthesis.
\end{itemize}
We can check if a statment is official or informal if it has the same number of brackets as binary connectors in the statement. If you see a bracket around a unitary connector, the sentence is not well-formed.
\textbf{Right Most Rule} We say the rightmost connector in a sentence with connectors of the same level the main connector.



\section{Truth Tables}
\subsection{Summary} First we remark that we are here talking about binary operators, 
so they will have and only have two operands. We have the following to consider:
\begin{itemize}
    \item \textbf{Logical OR ($\vee$)} is true when either one of the operands is true. 
        It evaluates to false otherwise.
    \item \textbf{Logical AND ($\land$)} is true when both of the operands are true and false otherwise.
    \item \textbf{Implication ($\implies$)} is true in two cases. The first case is when the
        first operand is false and second case is when both operands are true.
    \item \textbf{Double implication, Iff ($\iff$)} is true when either both operands are true 
        or when both operands are false.
    \item \textbf{Negation ($\neg$\footnote{Or in this course, we may see $\sim$})} is true when the operand
        is false and false otherwise.
\end{itemize}

\subsection{Full truth tables}
Example: \footnote{This example was adapted from Scharer 4.4 EG3.}
evaluate $(P\land \neg Q)\vee R. ~~\neg R\vee Q. ~~\therefore \neg P\implies Q$. 
Notice that this is equivalent to evaluating
\begin{equation*}
    \left(\left(\left(P\land \neg Q\right)\vee R \right) \land 
    \left(\neg R\vee Q\right)\right) \implies \neg P\implies Q
\end{equation*}
and by staring at it we see this statement is valid and only valid
 when $(P,Q,R)\in \{(T,T,T),(T,F,F),(F,T,T)\}$, which has a non-null 
 set of solution. Hence the statement is consistent.
\end{document}

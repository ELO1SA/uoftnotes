% Meta data
\documentclass[oneside, 12pt]{book}
\usepackage[utf8x]{inputenc}
\usepackage[english]{babel}
\usepackage{amsfonts}
\usepackage{amsmath}
\usepackage{url}
\usepackage{hyperref}

% Environment, new commands
\newcommand{\settag}[1]{\renewcommand{\theenumi}{#1}}
\newcommand{\R}{\mathbb{R}}
\newcommand{\double}[1]{\mathbb{#1}} % Set to behave like that on word
\newcommand{\qed}{\hfill $\mathcal{Q}.\mathcal{E}.\mathcal{D}.$}
\newcommand{\tbf}[1]{\textbf{#1}}
\newcommand{\tit}[1]{\textit{#1}}

\title{%
  \textbf{MAT224 Linear Algebra}\\
  \large Definitions, Lemmas, Theorems, Corollaries \\
    and their related proofs}
\author{Tingfeng Xia}
\date{Winter 2019}

\begin{document}
\maketitle
\newpage % make a new page for actual contents, possibly a content table
\mbox{}
\vfill
by Tingfeng Xia \\


Materials in this booklet are based heavily on Prof. Nicolas Hoell's lectures as well as A COURSE IN LINEAR ALGEBRA by David B. Damiano and John B. Little\newline\newline
The course website could be found here:\newline \url{http://www.math.toronto.edu/nhoell/MAT224/}\newline
\newpage
\tableofcontents

\chapter{Vector Spaces}
\section{(Real) Vector Space}
\settag{1.1.1}
\begin{enumerate}
    \item{\textbf{Definition of real vector space:}} A real vector space is a set $V$ together with scalar
    \begin{enumerate}
        \item \textbf{Closure under vector addition:} an operation called vector addition, which for each pair of vectors $\vec{x}, \vec{y}\in V$ produces another vector in $V$ denoted $\vec{x} + \vec{y}$, (i.e. $\forall \vec{x}, \vec{y}\in V, \vec{x} + \vec{y} \in V$) and
        \item \textbf{Closure under scalar multiplication:} an operation called multiplication by a scalar (a real number), which for each vector $\vec{x}\in V$, an each scalar $c\in \mathbb{R}$ produces another vector in $V$ denoted $c\vec{x}$. (i.e. $\forall \vec{x}\in V, \forall c \in \mathbb{R}, c \vec{x} \in V$)
    \end{enumerate}
    Furthermore, the two operations must satisfy the following axioms:(important)
    %\begin{enumerate}
        %\item For all vectors \textbf{x, y}, and \textbf{z} $\in V$, %(\textbf{x}+\textbf{y})+\textbf{x} 
    %\end{enumerate}
    \begin{enumerate}
        \item $\forall \vec{x}, \vec{y}, \vec{z} \in V, (\vec{x} + \vec{y}) + \vec{z} = \vec{x} + (\vec{y} + \vec{z})$
        \item $\forall \vec{v}, \vec{y} \in V, \vec{x} + \vec{y} = \vec{y} + \vec{x}$
        \item $\exists \vec{0} \in V s.t. \forall \vec{x} \in V, \vec{x} + \vec{0} = \vec{x}$ (Note that this property is a.k.a existence of additive identity)
        \item $\forall \vec{x} \in V, \exists (-\vec{x}) \in V~s.t.~\vec{x} + (-\vec{x}) = \vec{0}$ (Note that this property is a.k.a existence of additive inverse)
        \item $\forall \vec{x}, \vec{y} \in V, c \in \mathbb{R}, c(\vec{x} + \vec{y}) = c\vec{x} + c\vec{y}$
        \item $\forall \vec{x} \in V, c,d \in \mathbb{R}, (c + d)\vec{x} = c\vec{x} + d\vec{x}$
        \item $\forall \vec{x} \in V, c,d \in \mathbb{R}, (cd)\vec{x} = c(d\vec{x})$
        \item $\forall \vec{x} \in V, 1\vec{x} = \vec{x}$
    \end{enumerate}
    \settag{1.1.6}
    \item \textbf{Propositions for a R-v.s.} Let $V$ be a vector space. Then 
    \begin{enumerate}
        \item The zero vector is unique. Note that it might not necessarily be actually the zero vector in $\mathbb{R}^n$ that we are somewhat used to use.
        \item $\forall \vec{x} \in V, 0\vec{x}=0$
        \item $\forall \vec{x} \in V, $ the additive inverse is unique. Note that it might not necessarily be actually just $(-1)$ times the vector in $\mathbb{R}^n$ that we are somewhat used to use.
        \item $\forall \vec{x} \in V,~\forall c\in \mathbb{R}, ~(-c)\vec{x}=~-(c\vec{x})$
    \end{enumerate}
\end{enumerate}

\section{Sub-spaces}
    \begin{enumerate}
        \settag{1.2.4}
        \item \textbf{Usual definition of subspace applied to functions in $C^0(\mathbb{R})$}. Note that by $C^n(\cdot)$ we mean the function in this set are all of $Class-n$. Let $f,g \in C^0(\mathbb{R)}, let c\in \mathbb{R}$. Then, 
        \begin{enumerate}
            \item $f+g\in C^0(\R)$, and
            \item $cf \in C(\R)$
        \end{enumerate}
        The proof of this lemma relies on limit theorems of calculus.
        
        \settag{1.2.6}
        \item \textbf{(Intuitive) definition of (vector) subspace} Let $V$ be a vector space and let $W\subseteq V$ be a subset. Then $W$ is a (vector) subspace if $W$ is a vector subspace itself under the operations of vector sum and scalar multiplication from $V$.
        
        \settag{1.2.8}
        \item \textbf{Quick check rule for a subspace.} Let $V$ be a vector subspace, and let $W$ be a \textbf{non empty} subset of $V$. Then $W$ is a subspace of $V$ if and only if $\forall \vec{x}, \vec{y}\in W,~\forall c\in \R$, we have $c\vec{x}+\vec{y}\in W$.
        \settag{1.2.9}
        \item \textbf{Remark on the necessary condition of non-emptiness of subspace.} According to the definition of vector space that we gave in 1.1.1, a vector space must contain an additive identity element, hence it is necessary that we ensure $W\subseteq V$(from 1.2.6) is not an empty set.
        
        \settag{1.2.13}
        \item \textbf{Theorem: Intersection of sub-spaces is a subspace.} Let $V$ be a vector space. Then the intersection of any collection of sub-spaces of $V$ is a subspace of $V$.
        
        \settag{1.2.14}
        \item \textbf{Corollary: Hyper planes in $\R^n$ are sub-spaces of $\R^n$.} Let $a_{ij}(1\leq i\leq m)$, let $W_i = \{(x_1, ..., x_n)\in \R^n|a_{i1}x_1 + ... + a_{in}x_n = 0,~\forall 1 \leq i \leq m\}$. Then $W$ is a subspace of $\R^n$.
        
    \end{enumerate}
    
\section{Linear Combinations}
    \begin{enumerate}
        \settag{1.3.1}
        \item \textbf{Definitions regarding L.C. and derived spans.} Let $S$ be a subset of a vector space $V$, that is $S\subseteq V$.
        \begin{enumerate}
            \item a \textit{linear combination} of vectors in $S$ is any sum $a_1\vec{x}_1 + ... + a_n\vec{x}_n$, where the $a_i \in \R$, and the $x_i \in S$.
            \item we define the $Span$ of a set of vectors as follows to consider the special case of $S\stackrel{?}{=}\emptyset \in V$.
            \newline \underline{\tit{Case1: $S\neq \emptyset$:}} In this case, we define $Span(S)$ to be all possible linear combinations using vectors in $S$.\newline
            \underline{\tit{Case2: $S= \emptyset$:}} In this case, we define $Span(S = \emptyset)=\{\vec{0}\}$
            
            \item If $W=Span(S)$, we say $S$ \textit{spans(\text{or} generates)} $W$.
        \end{enumerate}
        
        \settag{1.3.4}
        \item \textbf{Span of a  subset of a vector space is a subspace.} Let $V$ be a vector space and let $S$ be any subset of $V$. Then $Span(S)$ is a subspace of $V$.
        
        \settag{1.3.5}
        \item \textbf{Sum of sets(with application to subs-paces).} Let $W_1\wedge W_2$ be sub spaces of a vector space $V$. The sum of $W_1$ and $W_2$ is the set
        \begin{center}
            $W_1+W_2 :=\{\vec{x}\in V |\vec{x}=\vec{x_1}+\vec{x_2}, \text{for some } \vec{x_1}\in W_1, \vec{x_2} \in W_2\}$
        \end{center}
        We think of the sum of the two sub-spaces(the two sets) as the set of vectors that can be built up from the vectors in $W_1$ and $W_2$ by linear combinations. Conversely, the vectors in the set $W_1+W_2$ are precisely the vectors that can be broken down into the sum of a vector in $W_1$ and a vector in $W_2$. One may find it helpful to view this as an analogue to a Cartesian product of the two set with a new constraint on the result.
        
        \settag{1.3.6}
        \item \textbf{Example.} If $W_1 = \{(a_1, a_2)\in \R^2|a_2 = 0\}$ and $W_2 \{(a_1, a_2)\in \R^2|a_1 = 0\}$, then $W_1 + W_2= \R^2$, since every vector in $\R^2$ can be written as the sum of vector in $W_1$ and a vector in $W_2$. For instance, we have $(5, -6)=(0, 5)+(0, -6)$, and $(5, 0)\in W_1 \wedge (0, -6)\in W_2$
        
        
        \settag{1.3.8} 
        \item \textbf{Proposition: The sum of spans of sets is the span of the union of the sets.} Let $W_1 = Span(S_1)$ and $W_2 = Span(S_2)$ be sub-spaces of a\textit{(the same)} vector space $V$. Then $W_1 + W_2 = Span(S_1 \cup S_2)$. Notice that the proof of this gave the important idea of mutual inclusion in proving sets are equal to each other.
        
        \settag{1.3.9}
        \item \textbf{The sum of sub-spaces is also a subspace.} Let $W_1$ and $W_2$ be sub-spaces of a vector space $V$. Then $W_1 + W_2$ is also a subspace of $V$.\newline
        \underline{\textit{Proof:}}\newline
        It is clear that $W_1 + W_2$ is non-empty, since neither $W_1$ nor $W_2$ is empty. Let $\vec{X}, \vec{y}$ be two vectors in $W_1+W_2$, let $c\in \R$. By our choice of $\vec{x}\text{and } \vec{y}$, we have
        \begin{align*}
            c\vec{x} + \vec{y} & = c(\vec{x}_1 + \vec{x}_2) + (\vec{y_1} + \vec{y_2}) \\
            & = (c\vec{x}_1 + \vec{y}_1) + (c\vec{x}_2 + \vec{y}_2) \in W_1+W_2
        \end{align*}
        Since $W_1$ and $W_2$ are sub-spaces of $V$, we have $(c\vec{x}_1 + \vec{y}_1) \in W_1\wedge (c\vec{x}_2 + \vec{y}_2)\in W_2$. Then by (1.2.8), we see that indeed $W_1 + W_2$ is a subspace of $V$. \qed
        
        \settag{1.3.10}
        \item \textbf{Remark.} In general =, if $W_1$ and $W_2$ are sub-spaces of $V$, then $W_1 \cup W_2$ will not be a subspace of $V$. For example, consider the two sub-spaces of $\R^2$ given in example (1.3.6). In that case $W_1\cup W_2$ is the union of two lines through the origin in $\R^2$. 
        
        \settag{1.3.11}
        \item \textbf{Proposition.} Let $W_1$ and $W_2$ be sub-spaces of vector space $V$ and let $W$ be a subspace of $V$ such that $W\supseteq W_1 \cup W_2$, then $W\supseteq W_1 + W_2$. Informally speaking, this proposition saying: "$W_1+W_2$ is the smallest subspace containing $W_1\cup W_2$", i.e., Any subspace that contains $W_1\cup W_2$ must be a super set of $W_1 + W_2$. \newline
        \tit{\underline{Proof:}}\newline
        We want to show: $W\supseteq W_1\cup W_2\Longrightarrow W\supseteq W_1 + W_2$\newline
            \begin{align*}
                \text{Assume that } \text{$W\supseteq W_1 \cup W_2$. Let $w_1\in W_1,~w_2\in W_2$.}\\
                \text{We notice that } w_1,~w_2\in W_1\cup W_2 \subseteq W \\
                \implies& w_1, w_2\in W \\
                \text{(Since $W$ is a subspace, so it is closed under addition)}\\
                \implies& w_1 + w_2 \in W\\
                \implies W_1 + W_2 \subseteq W \iff& W \supseteq W_1 + W_2 
            \end{align*}
            \qed
        
    \end{enumerate}

\section{Linear (In)dependence}
    \begin{enumerate}
        \settag{1.4.2}
        \item \textbf{Algebraic definition of linear dependence.} Let $V$ be a vector space, and let $S\subseteq V$.
            \begin{enumerate}
                \item A \textit{linear dependence} among the vectors of $S$ is an equation $a_1\vec{x} + ... + a_n\vec{x}_n = \vec{0}$ where the $x_i\in S$, and the $a_i\in \R$ are not all zero(i.e., at least one of the $a_i\neq 0$). In familiar\footnote{Familiar from MAT223, Prof. Jason Siefken's IBL(Inquiry Based Learning) notes.} words, there exists a non-trivial solution to the equation mentioned above.
                \item the set $S$ is said to be \textit{linearly dependent} if there exists a linear dependence among the vectors in $S$.
            \end{enumerate}
            \tbf{Remark: }It can be shown that the geometric\footnote{A set of vectors is said to be dependent of each other there exists a vector in this set, that it is in the Span of all other vectors in the set. I.e., There is some vectors in this set that are "redundant", it's position can be taken by some linear combination of the other vectors in the set.} definition and this are, in-fact, equivalent to each other. I will now produce the proof. \newline
        \tit{\underline{Proof of equivalence of definitions:}} \newline
        Let $V$ be a vector space, and let $S\subseteq V$. Consider the following equation:
        \begin{align*}
            a_1\vec{x}_1 + a_2\vec{x}_2 + ... + a_n\vec{x}_n &= 0\text{, where } \exists a_i\neq 0 \\
            \text{(WLOG, assume that } a_n &\neq \vec{0}\text{)}\\
            \implies \frac{a_1\vec{x}_1 + a_2\vec{x}_2 + ... + a_n\vec{x}_n}{a_n}&=\vec{0}\\
            \implies \vec{x}_n &= -\sum_{i=1}^{n-1}a_i\vec{x}_i
        \end{align*}
        Notice that the result $\vec{x}_n$ is in terms of all the other $(n-1)$ vectors in the set, hence a linear combination of those vectors, and this completes the proof.
        \qed
        
        \settag{1.4.4}
        \item \textbf{Algebraic definition of linear independence.} Let $V$ be a vector space, and $S\subseteq V$. Then $S$ is \textit{linearly independent} if whenever we have $a_i\in \R$ and $x_i\in S$ such that $a_1\vec{x}_1+...+a_n\vec{x}_n = \vec{0}$, then $a_i = 0,~\forall i$. A more conceivable way to understand this is if the aforementioned equation exists and only exists a set of trivial solution then the vectors involved in the equation are \textit{linearly independent}. 
        
        \settag{1.4.7}
        \item \textbf{Propositions regarding linear (in)dependency.}
            \begin{enumerate}
                \item Let $S$ be a linearly dependent subset of a vector space $V$, and let $S'$ be another subset of $V$ that contains $S$. Then $S'$ is also linearly dependent.
                \item Let $S$ be a linearly independent subset of vector space $V$ and let $S'$ be another subset of $V$ that is contained in $S$. Then $S'$ is also linearly independent.
            \end{enumerate}
        \underline{\tit{Proof of (a):}}
        Since $S$ is linearly dependent, there exists a linear dependence among the vectors in $S$, say, $a_1\vec{x}_1 + ...+ a_n\vec{x}_n = \vec{0}$. Since $S$ is contained in $S'$, this is also a linear dependence among the vectors in $S'$. Hence $S'$ is linear dependent.\qed\newline
        \underline{\tit{Proof of (b):}}
        Consider any equation $a_1\vec{x}_1 + ...+ a_n\vec{x}_n = \vec{0}$, where the $a_i \in \R,~\vec{x}_i \in S'$. Since $S'$ is contained in $S$, we can also view this as a potential linear dependence among vectors in $S$. However, $S$ is linearly independent, so it follows that all the $a_i = 0\in \R$. Hence $S'$ is also linearly independent.\qed
    \end{enumerate}
    
\section{Interlude on solving systems of linear equations}
    \begin{enumerate}
        \settag{1.5.1}
        \item \tbf{Definition}
    \end{enumerate}
    
\end{document}
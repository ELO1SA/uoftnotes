\documentclass{article}
\usepackage[utf8]{inputenc}
\usepackage{amsmath}
\usepackage{hyperref}

\title{Inferential Statistics, Test Statistics Manuel}
\author{Tingfeng Xia}
\date{2019, Winter Term}
\begin{document}
\maketitle
\tableofcontents
\newpage
\section{Test for $\mu = \mu_0$, w/$\sigma^2$ known}
Assume that $X_i\sim N(\mu, \sigma^2)$ are i.i.d, then the test 
statistic is
\begin{equation*}
    T(X) = \frac{\Bar{X} - \mu_0}{\sigma / \sqrt{n}} \sim N(0, 1)
\end{equation*}
then under $\alpha$ significance level, we have the rejection region
\begin{equation*}
    R_\alpha(T) = ( -\infty, z_{\frac{\alpha}{2}} ) \cup 
(z_{1-\frac{\alpha}{2}}, \infty )
\end{equation*}

\section{Test for $\mu = \mu_0$, w/$\sigma^2$ unknown}
Assume that $X_i\sim N(\mu, \sigma^2)$ are i.i.d, then the test 
statistic is
\begin{equation*}
    T(X) = \frac{\Bar{X} - \mu_0}{S / \sqrt{n}} \sim t_{n-1}
\end{equation*}
then under $\alpha$ significance level, we have the rejection region
\begin{equation*}
    R_\alpha(T) = ( -\infty, t_{\frac{\alpha}{2}, df=n-1} ) \cup 
(t_{1-\frac{\alpha}{2}, df=n-1}, \infty )
\end{equation*}

\section{Test for $\sigma^2 = \sigma_0^2$}
Assume that $X_i\sim N(\mu, \sigma^2)$ are i.i.d, then the test 
statistic is
\begin{equation*}
    T(X) = \frac{(n-1)S^2}{\sigma_0^2} \sim \chi^2_{df = n-1}
\end{equation*}
and the $\alpha$ significance level rejection region is 
\begin{equation*}
    R_\alpha(T) = ( -\infty, \chi^2_{\frac{\alpha}{2}, df=n-1} ) \cup 
(\chi^2_{1-\frac{\alpha}{2}, df=n-1}, \infty )
\end{equation*}

\section{Equality of Variances $\sigma_x = \sigma_y$}
If we have $X_1,\ldots,X_n\sim N(\mu_x, \sigma^2_x)$ and $Y_1,\ldots,Y_n\sim 
N(\mu_y, \sigma^2_y)$, then under our null hypothesis
\begin{equation*}
    T(X,Y) = \frac{S_x^2/\sigma_x^2}{S_y^2/\sigma_y^2} = \frac{S_x^2}{S_y^2} \sim F_{(n-1)(m-1)}
\end{equation*}
With $\alpha$ significance level, we then have the rejection region
\begin{equation*}
     R_\alpha(T) = \left(-\infty,F_{\frac{\alpha}{2}(n-1)(m-1)}\right) 
\cup \left(F_{1-\frac{\alpha}{2}(n-1)(m-1)}, \infty\right) 
\end{equation*}

\section{Equality of $\mu_x = \mu_y$, w/$\sigma_x, \sigma_y$ known}
If we have $\Bar{X} \sim N\left( \mu_x, \frac{\sigma_x^2}{n} \right)$ 
and $\Bar{Y} \sim N\left( \mu_y, \frac{\sigma_y^2}{n} \right)$, then 
\begin{equation*}
    T(X,Y) = \frac{\Bar{X} - \Bar{Y}}{\sqrt{\frac{\sigma_x^2}{n} + 
\frac{\sigma_y^2}{m}}}\sim N(0, 1)
\end{equation*}

\section{Equality of $\mu_x = \mu_y$, w/$\sigma=\sigma_x=\sigma_y$ 
known}
If this is the case, we can pull the $\sigma_x = \sigma_y = \sigma$ 
out from the above equation, we will have
\begin{equation*}
    T(X,Y) = \frac{\bar{X} - \bar{Y}}{\sigma \sqrt{(\frac{1}{n} + 
\frac{1}{m})}} \sim N(0, 1)
\end{equation*}

\section{Equality of $\mu_x = \mu_y$, w/$\sigma = \sigma_x = \sigma_y$ 
unknown}
If we have $\Bar{X} \sim N\left( \mu_x, \frac{\sigma_x^2}{n} \right)$ 
and $\Bar{Y} \sim N\left( \mu_y, \frac{\sigma_y^2}{n} \right)$, then 
\begin{equation*}
    T(X,Y) = \frac{\Bar{X} - \Bar{Y}}{S_p \sqrt{(\frac{1}{n} + 
\frac{1}{m})}} \sim t_{n+m-2}
\end{equation*}
where, $S_p$ is the polled sample variance, defined as
\begin{equation*}
    S_p^2 := \frac{(n-1)S_x^2 + (m-1)S_y^2}{n+m-2}
\end{equation*}

\section{Equality of $\mu_x = \mu_y$, w/$\sigma_x, \sigma_y$ unknown}
In this case, we have a messy formula for the degrees of freedom, the
test statistics that we use stays the same as above.

\section{Equality of $\mu_x = \mu_y$ for paired data}
We set the $H_0: \mu_x - \mu_y = 0$, define $D = X-Y$, then $\mu_d = \mu_x - \mu_y$. Notice that $\mu_d = 0 \iff \mu_x = \mu_y$, then our test statistic is
\begin{equation*}
    T(D) = \frac{\Bar{D}}{s_d/\sqrt{n}} \sim t_{n-1}
\end{equation*}

\section{Restricted Likelihood Ratio Test}
Define
\begin{equation*}
    \Lambda := \frac{max_{\theta \in 
\Omega_0}[L(\theta)]}{L(\Hat{\theta})}
\end{equation*}
Denoting $p = \dim\Omega =$ number of free var in the whole space, and 
$d = \dim\Omega_0 =$ number of free var under our null hypothesis, we 
have
\begin{equation*}
    T(X) = -2\ln{\Lambda} \xrightarrow{D} \chi^2_{df = p-d}
\end{equation*}

\section{Unrestricted Likelihood Ratio Test for Equality of $\mu_x = \mu_y$ for Normally Distributed Random Variables}
Consider i.i.d $X_1,\ldots,X_n \sim N(\mu_x, \sigma_x^2)$ and i.i.d $Y_1,\ldots,Y_m \sim N(\mu_y, \sigma_y^2)$. Notice that we have $p-d = 2 - 1 = 1$ in this case, and the likelihood is
\begin{equation*}
    L(\mu_x, \mu_y) = \left\{\left(2\pi \sigma_x^2\right)^{-\frac{n}{2}}e^{-\frac{1}{2\sigma_x^2}\sum_i{\left(X_i - \mu_x\right)^2}}\right\} \left\{\left(2\pi \sigma_y^2\right)^{-\frac{m}{2}}e^{-\frac{1}{2\sigma_y^2}\sum_i{\left(Y_i - \mu_y\right)^2}}\right\} \\
\end{equation*}    
by re-writing with $H_0: \mu = \mu_x = \mu_y$, we have
\begin{equation*}
    L(\mu) = \left\{\left(2\pi \sigma_x^2\right)^{-\frac{n}{2}}e^{-\frac{1}{2\sigma_x^2}\sum_i{\left(X_i - \mu\right)^2}}\right\} \left\{\left(2\pi \sigma_y^2\right)^{-\frac{m}{2}}e^{-\frac{1}{2\sigma_y^2}\sum_i{\left(Y_i - \mu\right)^2}}\right\}
\end{equation*}
Our test statistic is then
\begin{equation*}
    T(X,Y) = -2 \ln \Lambda = -2 \ln \frac{L(\hat{\mu})}{L(\hat{\mu_x}, \hat{\mu_y})} \sim \chi^2_{df = 1}
\end{equation*}

\section{Chi-Square Test of Goodness of Fit}
Suppose, $X_1, X_2, \ldots, X_k$ are the observed counts of category $1, 2, \ldots, k$ respectively. Then
\begin{equation*}
    (X_1, X_2, \ldots, X_k) \sim \text{Mult}(n, p_1, p_2, \ldots, p_k)~~~~\text{where}~E[X_i] = np_i (\geq 1), \forall i
\end{equation*}
and our test statistic will be, in this case
\begin{equation*}
    T(X) = X^2 = \sum_{i = 1}^k \frac{\left(X_i - np_i\right)^2}{np_i} \xrightarrow{D} \chi^2_{(df=k-1)}
\end{equation*}

\end{document}

